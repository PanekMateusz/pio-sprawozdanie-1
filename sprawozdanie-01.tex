\documentclass[10pt,a4paper]{article}
\usepackage[a4paper]{geometry}

\usepackage{polski}
\usepackage{xltxtra}

\usepackage{indentfirst}

\usepackage{fancyvrb}
\usepackage{relsize}
\usepackage[pdfborder={0 0 0}]{hyperref}

%% tweak fonts
\defaultfontfeatures{Mapping=tex-text}
\setromanfont{Charis SIL}
%\setsansfont[Scale=MatchLowercase]{Gill Sans}
%\setmonofont[Scale=MatchLowercase]{Menlo}
\linespread{1.25}

%% define custom commands and environments
\DefineVerbatimEnvironment%
  {SmallVerbatim}%
  {Verbatim}{fontsize=\relsize{-0.5},numbers=left,numbersep=-10pt,frame=lines,tabsize=4}

\begin{document}

%%fakesection{Tytuł}
\title{ 
  Sprawozdanie nr~1 z~laboratorium\\Podstaw Inżynierii Oprogramowania
}
\author{
  Grzegorz Bartkowiak\\
  Tomasz Cudziło\\
  Mateusz Ochtera\\
  Gustaw Wypych\\
  \\
  \textsc{PW EE Informatyka}\\[10pt]
}
\date{\today}

\maketitle

\section{Sytuacja biznesowa}

\section{Przygotowanie do projektu}
  \subsection{Skład zespołu}
    \begin{center}
    \begin{tabular}{ | l | r | }
       \hline
      \textbf{ Członek zespołu} &\textbf{ Stanowisko} \\ \hline \hline
       Adam Ktoś & Grafik/Webdesigner \\  \hline
       Zbysiu Innyktoś & Programista/Webdeveloper \\ \hline
       Czesław Niktgoniezna & Administrator baz danych \\ \hline
       John Tegonatomiastkazdyzna & Prawnik/Konsultant \\ \hline
     \end{tabular}
     \end{center}
  \subsection{Technologie}
  Jako grupa młodych informatyków, pomimo niewielkiego doświadczenia, jesteśmy doskonale zaznajomieni z najnowszymi  standardami W3C (dot. HTML, XHTML, CSS). Pośadamy również wiedzę na temat darmowego (Open Source) oprogramowania. Zredukuje to znacznie koszta naszego projektu, nie wpływając negatywnie na naszą efektowność.\\
\\
    \begin{itemize}
      \item Języki programowania : XHTML, CSS, PHP
      \item oprogramowanie: GIMP, Flash for Linux, Vim
      \item System kontroli wersji: GIT
    \end{itemize}

  \subsection{Wymagania w stosunku do kontrachentów}
    Firma ``Hammer.com'' udostępni, na swój koszt, zawodowego prawnika który, w roli konsultanta, pomoże nam ustalić legalność wprowadzanych opcji w interfejsie użytkownika. Sformułuje również regulamin na potrzeby serwisu, zgodny z polskim prawem ( może być wymagana translacja).  
  \subsection{Czas wykonania projektu}
    Wstępny projekt podglądowy (interfejs graficzny) zostanie wykonany w przeciągu tygodnia. Jeżeli zostanie zaakceptowany, przez kolejne dwa tygodnie zostaną napisane wszystkie funkcje/skrypty obsługujące stronę. Przewidziany jest dodatkowy tydzień opóźnienia na ewentualne poprawki. Do tego czasu zostaną wykonane liczne testy, aby zagwarantować niezawodność strony. Dalsza współpraca ze zleceniodawcą zależy od efektu naszej pracy i jego własnej decyzji.
  \subsection{Cena}
    Cały projekt został wyceniony na 8 000 PLN. Wszelkie poprawki wynikające z nieefektywnej funkcjonalności zobowiązani jesteśmy na własy koszt. Natiomiast zmiany interfejsu graficznego po zaakceptowaniu projektu wstępnego będą dodatkowo płatne.
    
    
\section{Zespół projektowy, zadania członków zespołu}

\section{Struktura podziału pracy}
\subsection{Macierz RAM}

\section{Struktura dokumentów}

\section{Wersjonowanie dokumentu}

\end{document}

